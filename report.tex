\documentclass[12pt]{article}

\usepackage{fullpage}
\usepackage{multicol,multirow}
\usepackage{tabularx}
\usepackage{ulem}
\usepackage[utf8]{inputenc}
\usepackage[russian]{babel}
\usepackage{minted}

\usepackage{color} %% это для отображения цвета в коде
\usepackage{listings} %% собственно, это и есть пакет listings

\lstset{ %
language=C,                 % выбор языка для подсветки (здесь это С)
basicstyle=\small\sffamily, % размер и начертание шрифта для подсветки кода
numbers=left,               % где поставить нумерацию строк (слева\справа)
%numberstyle=\tiny,           % размер шрифта для номеров строк
stepnumber=1,                   % размер шага между двумя номерами строк
numbersep=5pt,                % как далеко отстоят номера строк от подсвечиваемого кода
backgroundcolor=\color{white}, % цвет фона подсветки - используем \usepackage{color}
showspaces=false,            % показывать или нет пробелы специальными отступами
showstringspaces=false,      % показывать или нет пробелы в строках
showtabs=false,             % показывать или нет табуляцию в строках
frame=single,              % рисовать рамку вокруг кода
tabsize=2,                 % размер табуляции по умолчанию равен 2 пробелам
captionpos=t,              % позиция заголовка вверху [t] или внизу [b] 
breaklines=true,           % автоматически переносить строки (да\нет)
breakatwhitespace=false, % переносить строки только если есть пробел
escapeinside={\%*}{*)}   % если нужно добавить комментарии в коде
}

% Оригиналный шаблон: http://k806.ru/dalabs/da-report-template-2012.tex

\begin{document}
\begin{titlepage}
\begin{center}
\textbf{МИНИСТЕРСТВО ОБРАЗОВАНИЯ И НАУКИ РОССИЙСОЙ ФЕДЕРАЦИИ
\medskip
МОСКОВСКИЙ АВЦИАЦИОННЫЙ ИНСТИТУТ
(НАЦИОНАЛЬНЫЙ ИССЛЕДОВАТЬЕЛЬСКИЙ УНИВЕРСТИТЕТ)
\vfill\vfill
{\Huge ЛАБОРАТОРНАЯ РАБОТА №2} 
по курсу объектно-ориентированное программирование
I семестр, 2019/20 уч. год}
\end{center}
\vfill

Студент \uline{\it {Артемьев Дмитрий Иванович, группа М8О-206Б-18}\hfill}

Преподаватель \uline{\it {Журавлёв Андрей Андреевич}\hfill}

\vfill
\end{titlepage}

\subsection*{Условие}

Задание №1: написать класс, который реализует комплексные числа в алгебраической форме с операциями: 
\begin{enumerate}
\item сложения add, (a, b) + (c, d) = (a + c, b + d);
\item вычитания sub, (a, b) – (c, d) = (a – c, b – d);
\item умножения mul, (a, b) * (c, d) = (ac – bd, ad + bc);
\item деления div, (a, b) / (c, d) = (ac + bd, bc – ad) / (c2 + d2);
\item сравнение equ, (a, b) = (c, d), если (a = c) и (b = d);
\item сопряженное число conj, conj(a, b) = (a, –b);
\item сравнения модулей.
\end{enumerate}

\subsection*{Описание программы}

Исходный код лежит в 3 файлах:
\begin{enumerate}
\item src/main.cpp: основная программа, которая считывает 2 комплексных числа и обрабатывает их 
\item include/Complex.hpp: описание класса, объявление методов операций
\item src/Complex.cpp: реализация объявленных функций
\end{enumerate}

\subsection*{Дневник отладки}

Сравнении типа double нужно проводить с определённой точностью

\subsection*{Недочёты}

Не идеально прописаны файлы meson.build.

\subsection*{Выводы}

Я изучил перегрузку операторов в языке C++, систему сборки проектов meson, библиотеку для модульного тестирования Google C++ Testing Framework.

\vfill

\subsection*{Исходный код}

{\Huge Complex.hpp}
\inputminted
    {C++}{include/Complex.hpp}
    \pagebreak

{\Huge Complex.cpp}
\inputminted
    {C++}{src/Complex.cpp}
    \pagebreak
    
{\Huge main.cpp}
\inputminted
    {C++}{src/main.cpp}
    \pagebreak    
\end{document}
