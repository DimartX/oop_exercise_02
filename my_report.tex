\documentclass[12pt]{article}

\usepackage{fullpage}
\usepackage{multicol,multirow}
\usepackage{tabularx}
\usepackage{ulem}
\usepackage[utf8]{inputenc}
\usepackage[T2A]{fontenc}
\usepackage[russian,english]{babel}
\usepackage{minted}

% Оригиналный шаблон: http://k806.ru/dalabs/da-report-template-2012.tex

\begin{document}
\begin{titlepage}
\begin{center}
\textbf{МИНИСТЕРСТВО ОБРАЗОВАНИЯ И НАУКИ РОССИЙСОЙ ФЕДЕРАЦИИ
\medskip
МОСКОВСКИЙ АВЦИАЦИОННЫЙ ИНСТИТУТ
(НАЦИОНАЛЬНЫЙ ИССЛЕДОВАТЬЕЛЬСКИЙ УНИВЕРСТИТЕТ)
\vfill\vfill
{\Huge ЛАБОРАТОРНАЯ РАБОТА №1} 
по курсу объектно-ориентированное программирование
I семестр, 2019/20 уч. год}
\end{center}
\vfill

Студент \uline{\it {Артемьев Дмитрий Иванович, группа М8О-206Б-18}\hfill}

Преподаватель \uline{\it {Журавлёв Андрей Андреевич}\hfill}

\vfill
\end{titlepage}

\subsection*{Условие}

Задание №1: написать класс, который реализует комплексные числа в алгебраической форме с операциями: 
\begin{enumerate}
\item сложения add, (a, b) + (c, d) = (a + c, b + d);
\item вычитания sub, (a, b) – (c, d) = (a – c, b – d);
\item умножения mul, (a, b)  ́ (c, d) = (ac – bd, ad + bc);
\item деления div, (a, b) / (c, d) = (ac + bd, bc – ad) / (c2 + d2);
\item сравнение equ, (a, b) = (c, d), если (a = c) и (b = d);
\item сопряженное число conj, conj(a, b) = (a, –b);
\item сравнения модулей.
\end{enumerate}

\subsection*{Описание программы}

Исходный код лежит в 2 файлах:
\begin{enumerate}
\item src/main.cpp: основная программа, которая считывает 2 комплексных числа и обрабатывает их 
\item include/complex.hpp: описание класса, объявление и реализация методов операций
\end{enumerate}

\subsection*{Дневник отладки}

Самое нудное в лабораторных - писать тесты и отчёты.

\subsection*{Недочёты}

В TeXe неудобно писать, нужно допилить шаблоны для org-mode.
Чекер разбит на 3 файла.
Недокументированный код.

\subsection*{Выводы}

Не стоит писать лабы ночью.

\vfill

\subsection*{Исходный код}

{\Huge Complex.hpp}
\inputminted
{C++}{include/Complex.hpp}
\pagebreak

{\Huge main.cpp}
\inputminted
{C++}{src/main.cpp}
\pagebreak

\end{document}
